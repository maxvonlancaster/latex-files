\documentclass[a4paper,11pt]{article}
\usepackage{cmap}
\usepackage[cp1251]{inputenc}
\usepackage[english, ukrainian, russian]{babel}
\usepackage[left=2cm,right=1.5cm,top=1cm,bottom=1cm]{geometry}
\usepackage{amssymb}
\usepackage{graphicx}

\begin{document}

\pagenumbering{gobble}


\begin{large}

\begin{center}
\section*{Application of generative functions to the problems of maximum chess arrangements of n figures.}
\end{center}

\medskip

\begin{center}
\textbf{Abstract}
\end{center}


\hrule height 1pt
\vskip 3pt \hrule

\medskip
\medskip


\section*{Generating functions.}

Given sequence $\{a_{n}\}\in \mathbb{R}$ the generative function for it is defined as: $f(x)=\sum\limits_{n=0}^{\infty}a_{n}x^{n}.$ Generative functions provide a great way to operate over sequences, solve recurrent series, understand different enumeration problems etc.

Lets take a look at a simple example of a famous Euler's problem, and how it can be solved using generating functions.

\textbf{Example 1.1.}
Generative functions can be used to solve the recurrent series. Take, for example, famous Fibonacci sequence: $a_{0}=1$, $a_{1}=1$, $a_{n} = a_{n-1}+a_{n-2}$. For it we can write generative function $f(x) = \sum\limits_{n=0}^{\infty}a_{n}x^{n} =a_{0} + a_{1}x + \sum\limits_{n=2}^{\infty}(a_{n-1} + a_{n-2})x^{n} = a_{0} + a_{1}x + \sum\limits_{n=2}^{\infty}a_{n-1}x^{n} + \sum\limits_{n=2}^{\infty}a_{n-2}x^{n} = 1 + x + \sum\limits_{n=0}^{\infty}a_{n}x^{n+1} - x + \sum\limits_{n=0}^{\infty}a_{n}x^{n+2} = 1 + x\sum\limits_{n=0}^{\infty}a_{n}x^{n} + x^{2}\sum\limits_{n=0}^{\infty}a_{n}x^{n} = 1 + xf(x) + x^{2}f(x)$. So, we got the following equality: $f(x) = 1 + xf(x) + x^{2}f(x)$. From this, we get the generating function for Fibonacci sequence: $f(x) = \frac{1}{1 - x - x^{2}}$.


\section*{Some results for certain types of functions.}



\section*{n queens problem and some approaches.}

The $n$ queens problem is the problem of placing $n$ chess queens on an $n\times n$ chessboard so that no two queens attack each other. Solutions exist for all natural numbers $n$ with the exception of $n = 2$ and $n = 3$. Although the exact number of solutions is only known for $n = 27$, the asymptotic growth rate of the number of solutions is approximately $(0.143 n)^{n}$.


\section*{Partial solution for some figures.}

Lets consider the following figure: partial bishop~--- the figure, that will only be attacking on one line. We will denote the amount of different ways we can place $k$ such figures on a board $n\times n$ by $b_{n}^{k}$.

In case of board $1\times 1$, we get the amount of possible allocations of one such bishop is $1$; so $b_{1}^{1} = 1$ and $b_{1}^{0} = 1$. 

Let's assume, that we have the amount of possible allocations of $k$ figures for the board $(n-1)\times (n-1)$. The board $n\times n$ will contain board $(n-1)\times (n-1)$ as it's part (consider lower part to the left, see pic.). If we place $k-i$ figures in the part $(n-1)\times (n-1)$, in that case in the remaining $2n-1$ squares we place the rest $i$ figures. But the remaining $2n-1$ slice has only $2n-1-(k-i)$ squares available (as the rest $(k-i)$ squares are under the attack from the already placed $(k-i)$ bishops). That implies the total amount of placements for the remaining slice as $C_{2n-1-(k-i)}^{i}$ (where $C_{n}^{k}=\frac{n!}{(n-k)!k!}$~--- amount of combinations from $n$ to $k$). From that we get, that there are in total $b_{n-1}^{k-i} \cdot C_{2n-1-(k-i)}^{i}$ ways to place $k-i$ figures on $(n-1)\times (n-1)$ part of the board and $i$ figures on the rest of the board.

By adding all these products for $i=0,1,...,k$ we get the total amount of placements of $k$ figures on the board $n\times n$: $b_{n}^{k}=\sum\limits_{i=0}^{k} b_{n-1}^{k-i} \cdot C_{2n-1-(k-i)}^{i}$.

This way, we have proven the following theorem:

\textbf{Theorem 1.} The amount of possible arrangements of $k$ partial bishops on the board $n\times n$ is given by the following recursive formula: $$b_{n}^{k}=\sum\limits_{i=0}^{k} b_{n-1}^{k-i} \cdot C_{2n-1-(k-i)}^{i} (1)$$

Lets calculate $b_{n}^{k}$ for the case of $k=2$.

\textbf{Theorem 2.} The amount of arrangements of two partial bishops on board $n\times n$ is: $$b_{n}^{2} = 1 - 10(n+1)+\frac{29}{2}(n+2)(n+1) + \frac{16}{3}(n+3)(n+2)(n+1) + \frac{1}{2} (n+4)(n+3)(n+2)(n+1).$$

\textit{Proof.} By the recursive formula $(1)$ we have:  $b_{n}^{2}=b_{n-1}^{2} + b_{n}^{1}(2n-2) + \frac{1}{2}b_{n}^{0}(2n-1)(2n-2)$. Considering, that $b_{n-1}^{0}=1$ and $b_{n-1}^{1}=(n-1)^{2}$ (there are always 1 way to place 0 figures on any board and $n^{2}$ ways to place $1$ figure on the board $n\times n$) we get $b_{n}^{2}=b_{n-1}^{2} + 2n^{3} - 4n^{2} + 3n -1$. 

Lets find the generating function for this recursive sequence, and after that, we will be able to find the solution to this sequence. 







So far, there still remains open problem on the maximum arrangements of $n$ queens on the board $n\times n$.


\end{large}
\end{document}


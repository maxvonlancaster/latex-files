\documentclass[12pt,twoside,draft,en]{bmzhart}
%
% The following pakages are already included:
% amsmath,amsfonts,amsthm,amssymb,latexsym
%
% Theorem-like environments:
% {theorem}     - Theorem
% {proposition} - Proposition
% {lemma}       - Lemma
% {corollary}   - Corollary
% {remark}      - Remark
% {definition}  - Definition
% {example}     - Example
% {examples}    - Examples
% and their analogues with * :
% {theorem*}, {proposition*} etc.


\author{Author 1 N.S., Author 2 N.S.}
\title{Title of the article in English}

\shorttitle{Short title of the article}
% for headlines
% (only if main title too long)

\enabstract{Author 1 N.S., Author 2 N.S.}%
{Title of the article}%
{An abstract in English. An abstract in English. An abstract in English.
An abstract in English. An abstract in English. An abstract in English.

An abstract in English. An abstract in English. An abstract in English.
An abstract in English. An abstract in English. An abstract in English.}
{keyword one, keyword two}

\uaabstract{Author 1 N.S., Author 2 N.S. (in Ukrainian)}%
{Title of the article (in Ukrainian)}%
{An abstract (in Ukrainian). It will be translated to Ukrainian by Editorial Team (for
those authors, who don't understand Ukrainian)}
{keyword one, keyword two}

\thanks{Information on some grant ...}

\subjclass{?????, ?????}
% 2010 Mathematics Subject Classification
% According to http://www.ams.org/msc/msc2010.html


\UDC{?????}
% Universal Decimal Classification (���)
% It will be indicated by Editorial Team
% (for those authors, who don't understand what is it)

\pyear{2014} % year
\volume{6}   % volume
\issue{1}    % issue
\pageno{10}  % number of the first page
\received{01.07.2013} % received date
%\revised{?} % revised date

\institute{Institution, City, Country (Author 1 N.S.)\\
Institution, City, Country (Author 2 N.S.)}

\email{mail1@domain.ua (Author 1 N.S.), mail2@domain.ua (Author 2 N.S.)}

\def\baselinestretch{1.1}

\begin{document}

\maketitle

%%% ----------------------------------------------------------------------

\section*{Introduction}
Text of introduction. Text of introduction. Text of introduction. Text of introduction.
Text of introduction. Text of introduction. Text of introduction. Text of introduction.

Text of introduction. Text of introduction. Text of introduction. Text of introduction.
Text of introduction. Text of introduction. Text of introduction. Text of introduction.


\section{Section with results}

Text of the section. Text of the section. Text of the section.
Text of the section.

Text of the section. Text of the section. Text of the section.
Text of the section.

\begin{proposition*}
Example of a proposition without a number.
\end{proposition*}

\begin{theorem}\label{theorem1}
Example of a numbered theorem.
\end{theorem}
\begin{proof}
Proof of the theorem. Proof of the theorem. Proof of the theorem.
\end{proof}

Other theorem-like environments (lemma, corollary, remark, definition, example)
may be made analogously.

\begin{corollary}
Text of the corollary. Text of the corollary.
Text of the corollary. Text of the corollary.
\end{corollary}


\begin{lemma}
Text of the lemma. Text of the lemma.
Text of the lemma. Text of the lemma.
Text of the lemma. Text of the lemma.
\end{lemma}


\begin{proposition}
Text of the proposition. Text of the proposition.
Text of the proposition. Text of the proposition.
\end{proposition}

\begin{remark}
Text of the remark. Text of the remark.
Text of the remark. Text of the remark.
\end{remark}


\begin{definition}
Text of the definition. Text of the definition.
Text of the definition. Text of the definition.
\end{definition}

\begin{example}
Text of the example. Text of the example.
Text of the example. Text of the example.
\end{example}


Reference to Theorem \ref{theorem1} should be made via the command \verb"\ref".
Here is an example of display style formula with number
\begin{equation}\label{eq1}
\cos^2\varphi+\sin^2\varphi=1
\end{equation}
and reference \eqref{eq1} to it (made via the command
\verb"\eqref"). Don't use plain numbers like \verb"(1)".

The examples of formulas without a number are following
\begin{equation*}
\int_a^b f(x)\,dx=F(b)-F(a)
\end{equation*}
or
\[
e^x=\sum_{n=0}^\infty \frac{x^n}{n!}.
\]

In multiline equations the symbols like
$=$, $+$, $-$, $\le$, $\ge$, $<$, $>$ etc.
should be in the next line without a duplicating in the previous one:
\begin{equation*}
\begin{split}
\sum_{n\in\mathbb Z_+}\big\langle\otimes^n(T_t'f)\mid q_n\big\rangle&=
\sum_{n\in\mathbb Z_+}\big\langle T_t'f\otimes\dots\otimes T_t'f\mid
\varphi\otimes\dots\otimes\varphi\big\rangle\\
&=
\sum_{n\in\mathbb Z_+}\big\langle \otimes^n f\mid
(\otimes^n T_t)(q_n)\big\rangle.
\end{split}
\end{equation*}


If you need to move some text no the next line please use
command \verb"\linebreak" and don't use operators \verb"\\"
or \verb"\newline".


\section{Reference style}
The command \verb"\cite" should be used to obtain the references (like this \cite{k1} or
like this \cite{k2,k3,k4}). Also see examples of references \cite{ke1,ke2,ke3}.
Do not use plain numbers like \verb"[1, 2]".

The titles of journals should be abbreviated to the style used by American
Mathematical Society (http://www.ams.org/msnhtml/serials.pdf).

You must make attention that some journals have translated versions. Article published in this journals must be referenced like this \cite{k5} (also see example \cite{ke5}). Article published in other languages must be referenced like this \cite{k6} (also see example \cite{ke6}). Also you must note the Digital
Object Identifier (DOI) of the article in the references.

All references should be cited within the text; otherwise, these references will
be removed. References must be listed in alphabetical order; the following reference
style should be used:

\begin{thebibliography}{999}

% *** BOOK
\bibitem{k1} Author1 A.A., Author2 B.B., Author3 C.C. Title of the book. PublishingHouse, City, Year.

% *** BOOK as a part of a series
\bibitem{k2} Author1 A.A., Author2 B.B., Author3 C.C. Title of the book. In: Editor1 A.A.,
Editor2 B.B. (Eds.) SeriesTitle, Number. PublishingHouse, City, Year.

% *** ARTICLE IN ENGLISH
\bibitem{k3} Author1 A.A., Author2 B.B., Author3 C.C. \emph{Title of the article}.
Title of the Journal Year, \textbf{Volume} (Number), PageF--PageL. doi:????????????

% *** CONFERENCE ABSTRACT
\bibitem{k4} Author1 A.A., Author2 B.B. Title of abstract. In: Editor1 A.A., Editor2 B.B.
(Eds.) Proc. of the Intern. Conf. ``Title of the Conference'', City, Country, Month
DateF--DateL, Year, PublishingHouse, City, Year, PageF--PageL.

% *** ARTICLE IN ENGLISH WHICH HAS TRANSLATION
\bibitem{k5} Author1 A.A., Author2 B.B., Author3 C.C. \emph {Title of the article}. Title of the Journal Year, \textbf{Volume} (Number), PageF--PageL. doi:???????????? (translation of Title of the Journal Year, \textbf{Volume} (Number), PageF--PageL. doi:???????????? (in Language))

% *** ARTICLE IN UKRAINIAN OR OTHER LANGUAGE
\bibitem{k6} Author1 A.A., Author2 B.B., Author3 C.C. \emph{Title of the article}.
Title of the Journal Year, \textbf{Volume} (Number), PageF--PageL. (in Language)

%%%%%%%%%%%%%%%%%%%%%%%%%%%%%


% *** BOOK
\bibitem{ke1} Dineen S. Complex analysis on infinite-dimensional spaces.
 Springer-Verlag, London, 1999.

% *** BOOK as a part of a series
\bibitem{ke2} Defant A., Floret K. Tensor norms and operator ideals.
In: Nachbin L. (Ed.) Mathematics Studies, 176.
North-Holland, Amsterdam, 1993.


% *** ARTICLE
\bibitem{ke3} Lopushansky O., Zagorodnyuk A. \emph{Representing measures
and infinite-dimensional holomorphy}. J. Math. Anal. \& App. 2007, \textbf{333} (2), 614--625.
doi:10.1016/j.jmaa.2006.09.035

% *** ARTICLE WHICH HAS TRANSLATION
\bibitem{ke5} Zagorodnuyk A.V., Mitrofanov M.A. \emph {An analog of Wiener�s theorem for
infinite-dimensional Banach spaces}. Math. Notes 2015, \textbf{97} (1-2), 179--189. doi:10.1134/S0001434615010204 (translation of Mat. Zametki 2015, 97 (2), 191--202. doi:10.4213/mzm9371 (in Russian))

% *** ARTICLE IN UKRAINIAN OR OTHER LANGUAGE
\bibitem{ke6}
Ptashnyk B.Yo., Il'kiv V.S., Kmit' I.Ya., Polishchuk V.M. Nonlocal boundary value problems
for partial differential equations. Naukova Dumka, Kyiv, 2002. (in Ukrainian)

\end{thebibliography}

\end{document}

\newpage

\begin{center}
\section*{Lab Assignment 1: Bootstrap}
\end{center}

Bootstrap is a popular open-source front-end framework that simplifies web design and development. Created by Twitter in 2011, it provides a collection of HTML, CSS, and JavaScript components to help developers quickly build responsive, mobile-first websites. Bootstrap�s components, like grids, buttons, forms, and navigation bars, make it easy to design clean and professional-looking websites without extensive custom CSS.

\textit{Responsive Design:} Bootstrap is built to ensure that your website looks good on any device. It includes a grid system that allows you to create layouts that adapt to different screen sizes.

\textit{Pre-built Components:} From buttons to carousels, Bootstrap offers a range of components you can use and customize without needing to build them from scratch.

\textit{Consistent Design:} With a standardized set of UI elements, Bootstrap ensures that your website has a cohesive look and feel.

\textit{Easy to Customize:} Bootstrap is customizable with simple overrides to its variables, allowing you to adapt its styling to fit your brand.

\medskip

\textbf{Lab execution steps}

\begin{enumerate}
\item \textbf{Choose the topic of your assignment project.}

Choose a topic of your assignment project. You may come up with your own idea for a web page, or choose from the list below:

\begin{itemize}
\item Online shop

\item Personal portfolio

\item Student union web page

\item Tourism buro

\item Information page about your hometown

\item Online school

\item Technology bulletin
\end{itemize}

You may use template git repository, generated by following a link, provided by your mentor, or create your own repository with `index.html` and `styles.css` files.

\item \textbf{Link Bootstrap from your project.}

Open index.html and add the following in head section of file:

\begin{lstlisting}
<link
    href="https://cdn.jsdelivr.net/npm/bootstrap@5.2.3/dist/css/bootstrap.min.css"
    rel="stylesheet"
    integrity="sha384-rbsA2VBKQhggwzxH7pPCaAqO46MgnOM80zW1RWuH61DGLwZJEdK2Kadq2F9CUG65"
    crossorigin="anonymous">
\end{lstlisting}

And this line in the end of file, after body section ends:

\begin{lstlisting}
<script
    src="https://cdn.jsdelivr.net/npm/bootstrap@5.2.3/dist/js/bootstrap.bundle.min.js"
    integrity="sha384-kenU1KFdBIe4zVF0s0G1M5b4hcpxyD9F7jL+jjXkk+Q2h455rYXK/7HAuoJl+0I4"
    crossorigin="anonymous">
</script>
\end{lstlisting}

\item \textbf{Add navigation}

Navigation available in Bootstrap share general markup and styles, from the base .nav class to the active and disabled states. Swap modifier classes to switch between each style.

Add bootstrap navigation to your project. Example:

\begin{lstlisting}
<ul class="nav nav-tabs">
  <li class="nav-item">
    <a class="nav-link active" aria-current="page" href="#">Active</a>
  </li>
  <li class="nav-item dropdown">
    <a class="nav-link dropdown-toggle" data-bs-toggle="dropdown" href="#"
    role="button" aria-expanded="false">Dropdown</a>
    <ul class="dropdown-menu">
      <li><a class="dropdown-item" href="#">Action</a></li>
      <li><a class="dropdown-item" href="#">Another action</a></li>
      <li><a class="dropdown-item" href="#">Something else here</a></li>
      <li><hr class="dropdown-divider"></li>
      <li><a class="dropdown-item" href="#">Separated link</a></li>
    </ul>
  </li>
  <li class="nav-item">
    <a class="nav-link" href="#">Link</a>
  </li>
  <li class="nav-item">
    <a class="nav-link disabled" href="#" tabindex="-1" aria-disabled="true">Disabled</a>
  </li>
</ul>
\end{lstlisting}

\item \textbf{Add grid system}

Bootstrap's grid system divides the page into rows and columns, making it easy to organize content. With the grid, you can create flexible, responsive layouts that rearrange as the screen size changes.

Example:

\begin{lstlisting}
<div class="container">
    <div class="row">
        <div class="col-md-6">Column 1</div>
        <div class="col-md-6">Column 2</div>
    </div>
</div>
\end{lstlisting}

\item \textbf{Add carousel}

The carousel is a slideshow for cycling through a series of content, built with CSS 3D transforms. It works with a series of images, text, or custom markup. It also includes support for previous/next controls and indicators.

Carousels don�t automatically normalize slide dimensions. As such, you may need to use additional utilities or custom styles to appropriately size content. While carousels support previous/next controls and indicators, they�re not explicitly required. Add and customize as you see fit.

The .active class needs to be added to one of the slides otherwise the carousel will not be visible. Also be sure to set a unique id on the .carousel for optional controls, especially if you�re using multiple carousels on a single page. Control and indicator elements must have a data-bs-target attribute (or href for links) that matches the id of the .carousel element.

Add the carousel component to your html file like this:

\begin{lstlisting}
<div id="carouselExampleControls" class="carousel slide" data-bs-ride="carousel">
  <div class="carousel-inner">
    <div class="carousel-item active">
      <img src="..." class="d-block w-100" alt="...">
    </div>
    <div class="carousel-item">
      <img src="..." class="d-block w-100" alt="...">
    </div>
    <div class="carousel-item">
      <img src="..." class="d-block w-100" alt="...">
    </div>
  </div>
  <button class="carousel-control-prev" type="button" data-bs-target="#carouselExampleControls" data-bs-slide="prev">
    <span class="carousel-control-prev-icon" aria-hidden="true"></span>
    <span class="visually-hidden">Previous</span>
  </button>
  <button class="carousel-control-next" type="button" data-bs-target="#carouselExampleControls" data-bs-slide="next">
    <span class="carousel-control-next-icon" aria-hidden="true"></span>
    <span class="visually-hidden">Next</span>
  </button>
</div>
\end{lstlisting}

Add references to the images you'd like to see in your project.

\item \textbf{Add buttons}

Bootstrap includes several predefined button styles, each serving its own semantic purpose, with a few extras thrown in for more control.

Add buttons to your project (like sign in, scroll up etc.) using bootstrap styling. Examples:

\begin{lstlisting}
<button type="button" class="btn btn-primary">Primary</button>
<button type="button" class="btn btn-secondary">Secondary</button>
<button type="button" class="btn btn-success">Success</button>
<button type="button" class="btn btn-danger">Danger</button>
<button type="button" class="btn btn-warning">Warning</button>
<button type="button" class="btn btn-info">Info</button>
<button type="button" class="btn btn-light">Light</button>
<button type="button" class="btn btn-dark">Dark</button>
<button type="button" class="btn btn-link">Link</button>
\end{lstlisting}

\item \textbf{Add form}

Add submit form to your project (e.g., a feedback form). Example

\begin{lstlisting}
<form>
<div class="mb-3">
  <label for="exampleInputEmail1" class="form-label">Email address</label>
  <input type="email" class="form-control" id="exampleInputEmail1"
  aria-describedby="emailHelp">
  <div id="emailHelp" class="form-text">We`ll never share your email with anyone else.
  </div>
</div>
<div class="mb-3">
  <label for="exampleInputPassword1" class="form-label">Password</label>
  <input type="password" class="form-control" id="exampleInputPassword1">
</div>
<div class="mb-3 form-check">
  <input type="checkbox" class="form-check-input" id="exampleCheck1">
  <label class="form-check-label" for="exampleCheck1">Check me out</label>
</div>
<fieldset class="mb-3">
  <legend>Radios buttons</legend>
  <div class="form-check">
    <input type="radio" name="radios" class="form-check-input" id="exampleRadio1">
    <label class="form-check-label" for="exampleRadio1">Default radio</label>
  </div>
  <div class="mb-3 form-check">
    <input type="radio" name="radios" class="form-check-input" id="exampleRadio2">
    <label class="form-check-label" for="exampleRadio2">Another radio</label>
  </div>
</fieldset>
<div class="mb-3">
  <label class="form-label" for="customFile">Upload</label>
  <input type="file" class="form-control" id="customFile">
</div>
<div class="mb-3 form-check form-switch">
  <input class="form-check-input" type="checkbox" id="flexSwitchCheckChecked" checked>
  <label class="form-check-label"
  for="flexSwitchCheckChecked">
  Checked switch checkbox input
  </label>
</div>
<div class="mb-3">
  <label for="customRange3" class="form-label">Example range</label>
  <input type="range" class="form-range" min="0" max="5" step="0.5" id="customRange3">
</div>
<button type="submit" class="btn btn-primary">Submit</button>
</form>
\end{lstlisting}

\item \textbf{Add cards}

A card is a flexible and extensible content container. It includes options for headers and footers, a wide variety of content, contextual background colors, and powerful display options.

Cards are built with as little markup and styles as possible, but still manage to deliver a ton of control and customization. Built with flexbox, they offer easy alignment and mix well with other Bootstrap components. They have no margin by default, so use spacing utilities as needed.

Below is an example of a basic card with mixed content and a fixed width. Cards have no fixed width to start, so they�ll naturally fill the full width of its parent element. This is easily customized with various sizing options.

Example:

\begin{lstlisting}
<div class="card" style="width: 18rem;">
  <img src="..." class="card-img-top" alt="...">
  <div class="card-body">
    <h5 class="card-title">Card title</h5>
    <p class="card-text">Some quick example text to build on the card title and make up
    the bulk of the card`s content.</p>
    <a href="#" class="btn btn-primary">Go somewhere</a>
  </div>
</div>
\end{lstlisting}

\item \textbf{Add dropdown}

Dropdowns are toggleable, contextual overlays for displaying lists of links and more. They�re made interactive with the included Bootstrap dropdown JavaScript plugin. They�re toggled by clicking, not by hovering; this is an intentional design decision.

Example:

\begin{lstlisting}
<div class="dropdown">
  <button class="btn btn-secondary dropdown-toggle"
  type="button"
  id="dropdownMenuButton1"
  data-bs-toggle="dropdown"
  aria-expanded="false">
    Dropdown button
  </button>
  <ul class="dropdown-menu" aria-labelledby="dropdownMenuButton1">
    <li><a class="dropdown-item" href="#">Action</a></li>
    <li><a class="dropdown-item" href="#">Another action</a></li>
    <li><a class="dropdown-item" href="#">Something else here</a></li>
  </ul>
</div>
\end{lstlisting}

\item \textbf{Add table}

Due to the widespread use of `<table>` elements across third-party widgets like calendars and date pickers, Bootstrap�s tables are opt-in. Add the base class .table to any `<table>`, then extend with our optional modifier classes or custom styles. All table styles are not inherited in Bootstrap, meaning any nested tables can be styled independent from the parent.

Example:

\begin{lstlisting}
<table class="table table-striped table-hover">
<tr>
    <th>Name</th>
    <th>Description</th>
    <th>Email</th>
  </tr>
  <tr>
    <td>Item 1</td>
    <td>descr</td>
    <td>example@email</td>
  </tr>
  <tr>
    <td>Item 2</td>
    <td>descr</td>
    <td>example@email</td>
  </tr>
</table>
\end{lstlisting}

\item \textbf{Fill your project with content}

Fill web page with content on the topic you chose. Provide media (images, gifs etc.)
Align items properly on your page.

\item \textbf{Push your changes to git repository}

You can push changes to remote either via your code editor (e.g. vs code, web storm etc.), via git console
or by just uploading your files to github repository.
\end{enumerate}





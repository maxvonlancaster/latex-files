

\newpage

\begin{center}
\section*{Lab Assignment 2: Basics of JQuery}
\end{center}



jQuery is a fast, small, and feature-rich JavaScript library designed to simplify HTML DOM manipulation, event handling, animation, and AJAX interactions. Created in 2006 by John Resig, jQuery quickly became one of the most popular JavaScript libraries because it allowed developers to write less code to accomplish more, especially when it came to cross-browser compatibility and complex JavaScript functions.

\textit{Cross-browser Compatibility:} jQuery was originally created to standardize JavaScript behavior across browsers, making it a reliable tool for writing code that works on Chrome, Firefox, Safari, and Internet Explorer.

\textit{Simplified Syntax:} With a concise syntax, jQuery allows developers to perform complex tasks, such as DOM manipulation and AJAX requests, with ease.

\textit{Wide Plugin Ecosystem:} jQuery has a large ecosystem of plugins and extensions, making it easy to add functionality like sliders, date pickers, and form validation.

\textbf{Some of the key concepts:}

- \textit{DOM:} Document Object Model, a structured representation of web page, organized in a tree-like structure. The DOM represents everything on a webpage: HTML elements (like `<div>`, `<p>`, `<button>`), their content, and even their relationships to each other (like parent-child or sibling relationships). The DOM allows us to programmatically interact with our webpage. Using JavaScript (or a library like jQuery), we can access, modify, add and remove elements.

- \textit{Method binding:} In JavaScript, binding is the process of associating a function with a specific object so that the function has access to that object's properties. Think of it as "locking in" a function to a specific context (or object).

\medskip
\textbf{Lab execution steps}

\begin{enumerate}
\item \textbf{Choose the topic of your assignment project.}

Choose a topic of your assignment project. You may come up with your own idea for a web page, or choose from the list below:

\begin{itemize}
\item Online shop

\item Messaging board

\item Personal todo list

\item List of movies to watch
\end{itemize}

As part of this project you will implement an item tracking functionality using jQuery. You may use template git repository, generated by following a link, provided by your mentor, or create your own repository with `index.html` and `script.js` files.

\item \textbf{Link jQuery from your project.}

Add the following script to your `index.html` file:

\begin{lstlisting}
<script
  src="
  https://ajax.googleapis.com/ajax/libs/jquery/3.7.1/jquery.min.js">
</script>
\end{lstlisting}

\item \textbf{Add item submition section.}

Add the following item submition section to index.html file:

\begin{lstlisting}
<div class="container">
	<h1>Item list</h1>
    <div>Tracked items: <label id="tracked"></label></div>
	<div class="input-group">
		<input id="item-input" type="text" placeholder="Add new item">
		<button onclick="addItem()">Add Item</button>
	</div>
	<ul id="todo-list"></ul>
</div>
\end{lstlisting}

\item \textbf{Add method for adding new items to the list.}

Add the following method, that would append new items to your list:

\begin{lstlisting}
let count = 0;
list = $("#todo-list");

function addItem(){
  let inputValue = $("#item-input").val().trim();

  let newToDoElement = $("<div></div>");
  newToDoElement.attr("id", "todo-"+count);

  let delButton = $("<button></button>");
  delButton.attr("id", count);
  delButton.text("delete me");
  delButton.click(deleteToDo.bind(this, count));

  newToDoElement.text(inputValue);
  newToDoElement.append(delButton);

  list.append(newToDoElement);
  $("#item-input").val("");
  count++;
}

function deleteToDo(){

}

function updateCounts() {

}
\end{lstlisting}

Here line \lstinline{let inputValue = $("#item-input").val().trim();}  retrieves value from item with id `item-input` and sets it in variable `inputValue`. JQuery selectors allow you to retrieve data from webpage based on id, class or name of the element just - like css selectors.

Line \lstinline{let newToDoElement = $("<div></div>");} creates new div element and line \\ \lstinline{newToDoElement.attr("id", "todo-"+i);} sets value to the attribute `id` of that element.

Then we create newbutton element \lstinline{let delButton = $("<button></button>");}, give it id attribute and a text `delete me` and bind it to function that would be called when the button is clicked with method `delButton.click(deleteToDo.bind(this, count))`.

On line \lstinline{newToDoElement.text(inputValue);} we set text to the new element and on `newToDoElement.append(delButton);` we append delete button to it.

Then we \lstinline{list.append(newToDoElement);} - append item to the list and increase the amount of items tracked.

And finally, we clear input with \lstinline{$("#item-input").val("");}.

\item \textbf{Add method for deleting items from the list.}

Now lets write the following code to the \lstinline{deleteToDo()} method:

\begin{lstlisting}
function deleteToDo(j){
  let toDo = $("#todo-"+j);
  toDo.remove();
  updateCounts();
}
\end{lstlisting}

Method \lstinline{.remove()} deletes item from webpage.

\item \textbf{Add item tracking functionality.}

Update \lstinline{addItem()} to include adding a checkbox for tracking item:

\begin{lstlisting}
function addItem(){
  let inputValue = $("#item-input").val().trim();

  let newToDoElement = $("<div></div>");
  newToDoElement.attr("id", "todo-"+count);

  let delButton = $("<button></button>");
  delButton.attr("id", count);
  delButton.text("delete me");
  delButton.click(deleteToDo.bind(this, count));

  let checkbox = $("<input>").attr("type", "checkbox");
  checkbox.attr("class", "track-checkbox");
  checkbox.change(updateCounts);

  newToDoElement.text(inputValue);
  newToDoElement.append(delButton);

  newToDoElement.append(checkbox);

  list.append(newToDoElement);
  $("#item-input").val("");
  count++;
}
\end{lstlisting}

and a method that would be called when item is selected:

\begin{lstlisting}
function updateCounts() {
  let trackedCountElement = $("#tracked")
  count = $('.track-checkbox').length;
  trackedCount = 0;

  Array.from($('.track-checkbox')).forEach(checkbox => {
      if (checkbox.checked) {
          trackedCount++;
      }
  });
  trackedCountElement.html(trackedCount);
}
\end{lstlisting}

Here we select a list of checkboxes with

\lstinline{$(`.track-checkbox`)}, and go over every element with \lstinline{forEach()} cycle (first casting the list as array type). Then, we increase our counter by one for every checkbox, that is checked. Finally, we set the count to the element on the webpage.

\item \textbf{Add additional jQuery functionality.}

There are many other jQuery methods that might help you in web development, here are just some of the most widely used:

- \verb".css()": Gets or sets the CSS style properties of an element.

Example:

\begin{lstlisting}
$('#box').css('color', 'blue');
\end{lstlisting}

- \verb".addClass()" / \verb".removeClass()" / \verb".toggleClass()": Adds, removes, or toggles CSS classes on an element.

Example:


\begin{lstlisting}
$('#box').addClass('active');
$('#box').removeClass('active');
$('#box').toggleClass('highlighted');
\end{lstlisting}


- \verb".removeAttr()": Removes the attribute of the element

Example:

\begin{lstlisting}
$('img').removeAttr('alt');
\end{lstlisting}


- \verb".append()" / \verb".prepend()": Inserts content inside an element. \verb".append()" adds content at the end, and \verb".prepend()" at the beginning.

Example:

\begin{lstlisting}
$('#list').append('<li>New item</li>');
$('#list').prepend('<li>First item</li>');
\end{lstlisting}

- \verb".after()" / \verb".before()": Inserts content outside an element. \verb".after()" adds content after the element, \verb".before()" adds it before.

Example:

\begin{lstlisting}
$('#box').after('<p>After the box</p>');
$('#box').before('<p>Before the box</p>');
\end{lstlisting}

- \verb".remove()" / \verb".empty()": \verb".remove()" deletes an element, while \verb".empty()" clears its content.

Example:

\begin{lstlisting}
$('#box').remove();  // Removes the element
$('#container').empty();  // Clears inner content
\end{lstlisting}

- \verb".hide()" / \verb".show()": Hides or shows elements.

Example:

\begin{lstlisting}
$('#box').hide();
$('#box').show();
\end{lstlisting}

- \verb".fadeIn()" / \verb".fadeOut()": Animates an element�s opacity to create a fade-in or fade-out effect.

Example:

\begin{lstlisting}
$('#box').fadeIn();
$('#box').fadeOut();
\end{lstlisting}

- \verb".slideUp()" / \verb".slideDown()":  Slides an element up or down to hide or show it.

Example:

\begin{lstlisting}
$('#box').slideUp();
$('#box').slideDown();
\end{lstlisting}

- \verb".on()": Attaches event handlers to elements for various events.

Example:

\begin{lstlisting}
$('#button').on('mouseenter', function() {
    alert('Mouse entered!');
});
\end{lstlisting}

- \verb".each()": Iterates over a jQuery collection.

Example:

\begin{lstlisting}
$('li').each(function(index) {
    console.log('Item ' + index + ': ' + $(this).text());
});
\end{lstlisting}

- \verb".animate()": Creates custom animations by changing CSS properties over time.

Example:

\begin{lstlisting}
$('#box').animate({ width: '200px', opacity: 0.5 });
\end{lstlisting}

- \verb".ajax()": Sends an asynchronous HTTP request (for example, to backend server or to file, located in your project space).

Example:

\begin{lstlisting}
$.ajax({
    url: 'data.json',
    method: 'GET',
    success: function(data) {
        console.log(data);
    }
});
\end{lstlisting}

- \verb".parent()" / \verb".children()" / \verb".find()": Traverses DOM elements to find related elements. \verb".parent()" selects the parent, \verb".children()" selects children, and \verb".find()" searches descendants.

Example:

\begin{lstlisting}
$('#child').parent();        // Selects parent of #child
$('#parent').children();     // Selects children of #parent
$('#parent').find('.item');  // Selects .item within #parent
\end{lstlisting}

- \verb".ready()": Runs code when the DOM is fully loaded.

Example:

\begin{lstlisting}
$(document).ready(function() {
    alert('DOM is ready!');
});
\end{lstlisting}

Choose some of these to append additional functionality to your project.

\item \textbf{Fill your project with content}

Fill web page with content on the topic you chose. Add meaning to your project by replacing the abstract concept of an item with your own entity (todo item, product in online shop, etc.). Provide custom styles. Provide media (images, gifs etc.) Align items properly on your page.

\item \textbf{Push your changes to git repository}

You can push changes to remote either via your code editor (e.g. vs code, web storm etc.), via git console
or by just uploading your files to github repository.
\end{enumerate}
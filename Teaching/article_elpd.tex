\documentclass[a4paper,12pt]{article}
\usepackage{cmap}
\usepackage[cp1251]{inputenc}
\usepackage[english, ukrainian, russian]{babel}
\usepackage[left=1cm,right=1cm,top=1cm,bottom=1cm]{geometry}
\usepackage{amssymb}
\usepackage{graphicx}

\begin{document}

\pagenumbering{gobble}

\begin{large}


\begin{center}
\section*{Enhancing Accessibility: Supporting neurodivergent students in Higher Education.}

Dorosh A., Ilika S., Melnyk V., Shkilniuk D.

\end{center}

\bigskip


\hrule height 1pt
\vskip 3pt \hrule
\medskip
\medskip

\medskip

Neurodivergence is the state of being neurodivergent, this is when someone thinks, communicates, and/or learns differently and significantly from the predominant population. Autism, ADHD and dyslexia are all forms of neurodivergence because the brain is wired a different way.

Dyslexia is a learning disorder characterized by difficulties with accurate and fluent word recognition, spelling, and decoding abilities. Individuals with dyslexia may struggle with reading comprehension, writing, and phonological processing. It's essential to recognize that dyslexia is not indicative of intelligence; many individuals with dyslexia possess above-average intellectual abilities and strengths in other areas, such as creativity, problem-solving, and visual thinking.

\textbf{Types of Dyslexia:}

\textit{Dyspraxia}, also known as developmental coordination disorder (DCD), is a neurological condition that affects coordination, movement, and motor skills. Students with dyspraxia often face challenges in academic settings due to difficulties with tasks such as handwriting, organization, and physical activities.

\textit{Dyscalculia} is a learning disorder characterized by difficulties with mathematical concepts, calculations, and number sense. Students with dyscalculia often struggle with basic arithmetic, understanding mathematical symbols, and grasping mathematical concepts.

\textit{Dysgraphia} is a learning disorder characterized by difficulties with writing, handwriting, and fine motor skills. Students with dysgraphia often struggle with letter formation, spacing, spelling, and organization of written work.


\textbf{Challenges Faced by Students with Dyslexia in Higher Education:}

In our investigation, we worked closely with two students diagnosed with dyslexia who faced challenges with traditional text-based learning materials. One student also displayed strong characteristics of Dyscalculia and Dyspraxia, even though official diagnosis of these particular types of Dyslexia is absent. Despite possessing strong cognitive abilities and a keen interest in the subject matter, the students struggled with processing written information efficiently and retaining key concepts. Recognizing the\\ importance of addressing these challenges, we sought to identify alternative approaches that would cater to the student's strengths and learning preferences.

Transitioning to higher education can be daunting for students with dyslexia due to the increased academic demands, complex reading materials, and writing-intensive assignments. Some common challenges faced by these students include:

\begin{itemize}
\item Reading Complex Texts: College-level textbooks and academic articles often contain dense, technical language and complex concepts, which can pose significant challenges for students with dyslexia.

\item Writing Assignments: Writing essays, research papers, and reports requires strong literacy skills, including spelling, grammar, and sentence structure, areas in which students with dyslexia may struggle.

\item Time Management: Students with dyslexia may require additional time to complete reading assignments, write papers, and study for exams due to the time-intensive nature of reading and writing tasks.

\item Note-Taking: Taking comprehensive notes during lectures can be challenging for students with dyslexia, particularly if they struggle with handwriting or spelling.

\item Exam Accommodations: Standardized exams and timed tests may not accurately assess the knowledge and abilities of students with dyslexia, necessitating accommodations such as extended time, alternative formats, or assistive technology.
\end{itemize}

\textbf{Choosing Visual Presentation of Material:}

After careful consideration and consultation with the student, we determined that a more visual presentation of learning materials would be beneficial. Visual learning strategies leverage images, diagrams, charts, and other visual aids to convey information, making abstract concepts more tangible and facilitating comprehension. By incorporating visual elements into course materials and instructional activities, we aimed to provide the student with alternative pathways to understanding complex concepts and enhancing learning outcomes.

\textbf{Implementation of Visual Learning Strategies:}

The implementation of visual learning strategies involved several key components:

\begin{itemize}
\item Visual Aids in Lectures: During lectures, instructors utilized visual aids such as slideshows, diagrams, and multimedia presentations to supplement verbal explanations. Visuals were carefully designed to illustrate key points, reinforce concepts, and provide context, thereby enhancing the student's understanding and engagement.

\item Visual Text Adaptations: Course materials, including textbooks, readings, and handouts, were adapted to include more visual elements such as illustrations, infographics, and annotated diagrams. Visual text adaptations were designed to break down information into digestible chunks, improve readability, and support comprehension for students with dyslexia.

\item Interactive Learning Activities: Incorporating interactive learning activities such as group discussions, collaborative projects, and hands-on exercises provided the student with opportunities to engage with course content in a dynamic and multisensory manner. Interactive activities encouraged active participation, fostered critical thinking skills, and facilitated deeper understanding of subject matter.

\item Access to Visual Technology Tools: The student was provided with access to visual technology tools and assistive technologies designed to support visual learning, such as mind mapping software, graphic organizers, and digital drawing applications. These tools empowered the student to create visual representations of concepts, organize information spatially, and express ideas creatively.
    
\end{itemize}

Additional approaches taken:

\begin{itemize}
    
\item Break Tasks into Manageable Steps: Break down complex tasks into smaller, more manageable steps to help students with neurodivergence approach tasks systematically. Provide clear instructions and demonstrate each step to ensure comprehension and promote success.
    
\item Positive Reinforcement: Provide positive reinforcement and encouragement to students with neurodivergence to build their confidence and motivation. Recognize and celebrate their efforts, progress, and achievements in writing, and create a supportive and inclusive classroom environment where all students feel valued and respected.
    
\item Organizational Support: Teach students with neurodivergence strategies for organizing their thoughts and structuring written work. Use graphic organizers, outlines, and templates to help students plan and organize their ideas before writing, and provide visual cues and reminders to reinforce organizational strategies.
    
\item Scaffolded Support: Offer scaffolded support to gradually build students' confidence and independence in mathematics. Provide guided practice, modeling, and feedback to help students develop problem-solving skills and strategies for overcoming challenges.
\end{itemize}

\textbf{System for sign language recognition.}

Members of the Mathematical Modeling Department at Chernivtsi National University took part in developing the system for  Sign language (SL) recognition. As a result, we have developed a training system for SL studing. Our system helps to study The dactyl alphabet and use it for gesture reproducing (for example terms).
SL  is a means of communication between persons with hearing difficulties among themselves and with other persons including those with normal hearing. SL uses visual-kinesthetic means (hand gestures, lip movements, and facial expressions of emotions). For example, the Ukrainian SL combines several versions of sign communications, namely, a dactyl alphabet for the representation of letters of the Ukrainian alphabet and also words, language constructions, and semantic concepts. The dactyl alphabet (expressing letters with the help of fingers) performs an auxiliary function (of representing names, denotations, etc.). In SL, a definite unique sign equivalent corresponds to each semantic concept.

Modern computer facilities and web technologies will make it possible to overcome these dialectical differences. To this end, modern technologies, namely, multimedia computer facilities are necessary for describing and reproducing SL. For the Ukrainian SL, such facilities are absent. Therefore, the creation of modern technologies of SL notation and reproduction is a topical problem.
We propose this system to colleagues from other departments, to study SL and for communication with deaf students.


\textbf{Conclusion}:
Dyslexia and other types of neurodivergence present unique challenges for students in the classroom, but with understanding, patience, and effective support strategies, teachers can help students with dysgraphia overcome obstacles, develop essential writing skills, and succeed academically. By implementing assistive technology, handwriting support, alternative writing methods, modified assessments, organizational support, and positive reinforcement, educators can create inclusive learning environments where all students have the opportunity to reach their full potential in written expression.

By embracing visual learning as a means of enhancing accessibility and inclusivity, educators can empower neurodivergent students to overcome learning barriers and achieve academic success. Through thoughtful implementation of visual-based educational strategies, colleges and universities can create learning environments that cater to the diverse needs of all learners, fostering a culture of equity, diversity, and inclusion in higher education.

Together, we can empower students with neurodivergence to build confidence, develop effective writing skills, and become proficient specialists.

\end{large}

\end{document}


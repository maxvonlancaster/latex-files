\documentclass[a4paper,12pt]{article}
\usepackage{cmap}
\usepackage[cp1251]{inputenc}
\usepackage[english, ukrainian, russian]{babel}
\usepackage[left=2cm,right=1.5cm,top=1cm,bottom=1cm]{geometry}
\usepackage{amssymb}
\usepackage{graphicx}
\usepackage{listings}
\usepackage{color}

\definecolor{dkgreen}{rgb}{0,0.6,0}
\definecolor{gray}{rgb}{0.5,0.5,0.5}
\definecolor{mauve}{rgb}{0.58,0,0.82}

\lstset{frame=tb,
  language=Java,
  aboveskip=3mm,
  belowskip=3mm,
  showstringspaces=false,
  columns=flexible,
  basicstyle={\small\ttfamily},
  numbers=none,
  numberstyle=\tiny\color{gray},
  keywordstyle=\color{blue},
  commentstyle=\color{dkgreen},
  stringstyle=\color{mauve},
  breaklines=true,
  breakatwhitespace=true,
  tabsize=3
}

\begin{document}

\pagenumbering{gobble}


\begin{large}

\begin{center}
\section*{SOLID.}
\end{center}

\medskip

\section*{What are SOLID Design Principles?}

SOLID is an acronym that stands for:

\begin{itemize}
\item Single Responsibility Principle (SRP)

\item Open-Closed Principle (OCP)

\item Liskov Substitution Principle (LSP)

\item Interface Segregation Principle (ISP)

\item Dependency Inversion Principle (DIP)
\end{itemize}

In the coming sections, we�ll look at what each of those principles means in detail.

The SOLID design principles are a subcategory of many principles introduced by the American computer scientist and instructor, Robert C. Martin (A.K.A Uncle Bob) in a 2000 paper.

Following these principles can result in a very large codebase for a software system. But in the long run, the main aim of the principles is never defeated. That is, helping software developers make changes to their code without causing any major issues.

\section*{The Single Responsibility Principle (SRP)}

The single responsibility principle states that a class, module, or function should have only one reason to change, meaning it should do one thing.

For example, a class that shows the name of an animal should not be the same class that displays the kind of sound it makes and how it feeds.

\textbf{Example}:

\begin{lstlisting}
public class BadBook {
    private String name;
    private String author;
    private String text;
    //...

    void printTextToConsole(){
        // our code for formatting and printing the text
    }
}
\end{lstlisting}

This code violates the single responsibility principle we outlined earlier.

To fix our mess, we should implement a separate class that deals only with printing our texts:

\begin{lstlisting}
public class BookPrinter {

    // methods for outputting text
    void printTextToConsole(String text){
        //our code for formatting and printing the text
    }

    void printTextToAnotherMedium(String text){
        // code for writing to any other location..
    }
}
\end{lstlisting}


\section*{The Open-Closed Principle (OCP)}

The open-closed principle states that classes, modules, and functions should be open for extension but closed for modification.

This principle might seem to contradict itself, but you can still make sense of it in code. It means you should be able to extend the functionality of a class, module, or function by adding more code without modifying the existing code.

\textbf{Example}:

\begin{lstlisting}
public class Guitar {

    private String make;
    private String model;
    private int volume;

    //Constructors, getters & setters
}
\end{lstlisting}

What if we later decide the Guitar is a little boring and could use a cool flame pattern to make it look more rock and roll.

At this point, it might be tempting to just open up the Guitar class and add a flame pattern � but who knows what errors that might throw up in our application.

Instead, let�s stick to the open-closed principle and simply extend our Guitar class:

\begin{lstlisting}
public class SuperCoolGuitarWithFlames extends Guitar {

    private String flameColor;

    //constructor, getters + setters
}
\end{lstlisting}


\section*{The Liskov Substitution Principle (LSP)}

The Liskov substitution principle is one of the most important principles to adhere to in object-oriented programming (OOP). It was introduced by the computer scientist Barbara Liskov in 1987 in a paper she co-authored with Jeannette Wing.

The principle states that child classes or subclasses must be substitutable for their parent classes or super classes. In other words, the child class must be able to replace the parent class. This has the advantage of letting you know what to expect from your code.

\textbf{Example}:

\begin{lstlisting}
public interface Car {

    void turnOnEngine();
    void accelerate();
}
\end{lstlisting}

Above, we define a simple Car interface with a couple of methods that all cars should be able to fulfill: turning on the engine and accelerating forward.

But suppose, we have an electric car:

\begin{lstlisting}
public class ElectricCar implements Car {

    public void turnOnEngine() {
        throw new AssertionError("I don't have an engine!");
    }

    public void accelerate() {
        //this acceleration is crazy!
    }
\end{lstlisting}

By throwing a car without an engine into the mix, we are inherently changing the behavior of our program. This is a blatant violation of Liskov substitution and is a bit harder to fix than our previous two principles.

One possible solution would be to rework our model into interfaces that take into account the engine-less state of our Car.

\begin{lstlisting}
public interface EnginelessCar {

    void accelerate();
}
\end{lstlisting}



\section*{The Interface Segregation Principle (ISP)}

The interface segregation principle states that clients should not be forced to implement interfaces or methods they do not use.

More specifically, the ISP suggests that software developers should break down large interfaces into smaller, more specific ones, so that clients only need to depend on the interfaces that are relevant to them. This can make the codebase easier to maintain.

This principle is fairly similar to the single responsibility principle (SRP). But it�s not just about a single interface doing only one thing � it�s about breaking the whole codebase into multiple interfaces or components.

Think about this as the same thing you do while working with frontend frameworks and libraries like React, Svelte, and Vue. You usually break down the codebase into components you only bring in when needed.

This means you create individual components that have functionality specific to them. The component responsible for implementing scroll to the top, for example, will not be the one to switch between light and dark, and so on.

\textbf{Example}:

\begin{lstlisting}
public interface BearKeeper {
    void washTheBear();
    void feedTheBear();
    void petTheBear();
}
\end{lstlisting}

As avid zookeepers, we�re more than happy to wash and feed our beloved bears. But we�re all too aware of the dangers of petting them. Unfortunately, our interface is rather large, and we have no choice but to implement the code to pet the bear.

Let�s fix this by splitting our large interface into three separate ones:

\begin{lstlisting}
public interface BearCleaner {
    void washTheBear();
}

public interface BearFeeder {
    void feedTheBear();
}

public interface BearPetter {
    void petTheBear();
}
\end{lstlisting}

\section*{The Dependency Inversion Principle (DIP)}

The dependency inversion principle is about decoupling software modules. That is, making them as separate from one another as possible.

The principle states that high-level modules should not depend on low-level modules. Instead, they should both depend on abstractions. Additionally, abstractions should not depend on details, but details should depend on abstractions.

In simpler terms, this means instead of writing code that relies on specific details of how lower-level code works, you should write code that depends on more general abstractions that can be implemented in different ways.

This makes it easier to change the lower-level code without having to change the higher-level code.

\textbf{Example}:

\begin{lstlisting}
public class Windows {

    private StandardKeyboard keyboard;
    private Monitor monitor;

    public Windows() {
        monitor = new Monitor();
        keyboard = new StandardKeyboard();
    }

}
\end{lstlisting}

Not only does this make our Windows class hard to test, but we�ve also lost the ability to switch out our StandardKeyboard class with a different one should the need arise. And we�re stuck with our Monitor class too.

Let�s decouple our machine from the StandardKeyboard by adding a more general Keyboard interface and using this in our class:

\begin{lstlisting}
public interface Keyboard { }

public class StandardKeyboard implements Keyboard { }

public class Windows{

    private Keyboard keyboard;
    private Monitor monitor;

    public Windows(Keyboard keyboard, Monitor monitor) {
        this.keyboard = keyboard;
        this.monitor = monitor;
    }
}
\end{lstlisting}





\end{large}
\end{document}


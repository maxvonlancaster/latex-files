
\pagebreak
\begin{center}
\textbf{Authentication, Authorization, JWT and OAuth2.0 .}
\end{center}

\medskip

\textbf{Authentication and authorization} are two critical components of application security. Together, they ensure that only the right users can access specific resources or functionalities within a system, protecting sensitive data and maintaining system integrity.

Authentication is the process of verifying the identity of a user or entity. It ensures that the individual attempting to access a system is who they claim to be.

Authorization is the process of determining what actions or resources an authenticated user is allowed to access. It controls the permissions and privileges granted to a user or entity.

Modern applications use a variety of tools and techniques to implement authentication and authorization securely:

\begin{itemize}
\item Session-Based Authentication: The server creates a session for the user upon login and stores it in memory. A session ID is sent to the client as a cookie.
    
\item Token-Based Authentication: After login, the server generates a token (e.g., JWT) that the client includes in each request for authentication.
    
\item Authorization can be implemented via middleware, Access Control Lists (ACLs) or API gateways.
\end{itemize}


\textbf{JSON Web Tokens}, commonly known as JWTs, are a compact and secure way of transmitting information between parties as a JSON object. This technology has become particularly popular in modern web applications due to its simplicity and robustness in enabling secure, stateless authentication and authorization.

What is JWT?
JWT is an open standard (RFC 7519) that defines a way to securely transmit information as a JSON object. The token itself is self-contained, meaning it holds all the information needed for authentication, which can be verified without querying a database. This capability makes JWT highly scalable, as it reduces the server's reliance on session storage.


A JWT consists of three parts, separated by dots (.):

\begin{itemize}
\item Header: Contains metadata about the token, including the type of token (JWT) and the hashing algorithm used (e.g., HMAC SHA256 or RSA).

\item Payload: Holds the claims, which are statements about the user or other data. Claims can be predefined, such as sub (subject) and exp (expiration), or custom, like role or email.

\item Signature: Created by encoding the header and payload using a secret key or a private key, depending on the algorithm.
\end{itemize}

An example JWT might look like this:

\begin{lstlisting}
eyJhbGciOiJIUzI1NiIsInR5cCI6IkpXVCJ9.
eyJzdWIiOiIxMjM0NTY3ODkwIiwibmFtZSI6IkpvaG4gRG9lIiwiaWF0IjoxNTE2MjM5MDIyfQ.
SflKxwRJSMeKKF2QT4fwpMeJf36POk6yJV_adQssw5c
\end{lstlisting}

Decoded, it would look like this:


\begin{lstlisting}
{
  "alg": "HS256",
  "typ": "JWT"
}
{
  "sub": "1234567890",
  "name": "John Doe",
  "iat": 1516239022
}
HMACSHA256(
  base64UrlEncode(header) + "." +
  base64UrlEncode(payload),

your-256-bit-secret
)
\end{lstlisting}

\textbf{OAuth 2.0}, short for Open Authorization 2.0, is a protocol that enables third-party applications to gain limited access to a user's resources without sharing their username and password. It provides an efficient way to implement secure delegated access, focusing on authorization rather than authentication.



\begin{lstlisting}

\end{lstlisting}





